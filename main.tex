% Template Διπλωματικών Εργασιών Πανεπιστημίου Δυτικής Μακεδονίας, Τμήμα Ηλεκτρολόγων Μηχανικών και Μηχανικών Υπολογιστών
%
% Αναπτύχθηκε από την Αναστασία Τερζή και το Μηνά Δασυγένη το 2021 για τη διευκόλυνση των φοιτητών που θέλουν να γράψουν τις διπλωματικές τους εργασίες στο LaTex
%
% https://github.com/mdasyg/tempate_diploma_UOWM_ECE.git
%
% Διορθώσεις/Βελτιώσεις στο template να τις στείλετε στο mdasyg@ieee.org ή να τις κάνετε push στο github.
%
% Άδεια: Αναφορά Δημιουργού - Μη Εμπορική Χρήση - Παρόμοια Διανομή CC BY-NC-SA
% 
% Δοκιμάστηκε με το xelatex 2020/2021
% Έκδοση: 2021/08

\documentclass[12pt, letterpaper]{book}
\usepackage[greek,english]{babel}
% \usepackage[utf8]{inputenc}
\usepackage[a4paper,width=150mm,top=25mm,bottom=25mm]{geometry}
\usepackage{blindtext}
\usepackage[acronym]{glossaries}
\makeglossaries
%to include all files. as subfiles
\usepackage{subfiles}
%customized footer
\usepackage{fancyhdr}
\pagestyle{fancy}
\fancyhf{}
\renewcommand{\headrulewidth}{0pt}
\renewcommand{\footrulewidth}{2pt}
\lfoot{Τιτλος Εργασίας}
\rfoot{\thepage}
%for the line under section name
\usepackage{tocloft}
\renewcommand\cftaftertoctitle{\par\noindent\hrulefill\par\vskip 1.3em}

\usepackage[toc,page]{appendix}
\renewcommand\appendixtocname{Παραρτήματα}





\usepackage{parskip}
\usepackage{titlesec}
\usepackage{afterpage}
\usepackage{graphicx}
\usepackage{xcolor}

\usepackage{polyglossia}
\usepackage[LGR,T1]{fontenc}
%play font included
\setromanfont{Play}[
    Path=./Play/,
    Extension = .ttf ,
    UprightFont =*-Regular,
    BoldFont =*-Bold
    ]

% Define Illustrations environment
\usepackage{newfloat}
\DeclareFloatingEnvironment{illustration}
\renewcommand\listillustrationname{ \LARGE Κατάλογος Εικόνων}
      

\makeatletter
\renewcommand{\@chapapp}{Κεφάλαιο}
\makeatother
      


\begin{document}
%for the empty pages after title page
\newcommand\blankpage{
    \null
    \thispagestyle{empty}
    \addtocounter{page}{-1}
    \newpage
    }
    
%Εμπροσθόφυλλο Ελληνικά
\begin{titlepage}
 \includegraphics[width=0.6\textwidth]{images/logo}

    \begin{center}
        \vspace*{3cm}
            
        \Huge
        \color{orange}
        \textbf{Τίτλος Διπλωματικής}\\
        \color{black}  
        \vspace{0.5cm}
        \LARGE
        Υπότιτλος\\
        \vspace{0.5cm}
        \small
        του/της\\
            
        \vspace{1.5cm}
        \LARGE    
        \textbf{Όνομα Συγγραφέα}
            
         \vspace{3.5cm}
        \textbf{Επιβλέπων:} Όνομα και Τίτλος καθηγητή
        \vfill    
        \vspace{0.8cm}
            
       
            
        \Large
       ΚΟΖΑΝΗ/ΜΗΝΑΣ/ΕΤΟΣ
       \end{center}

\end{titlepage}
%για να αποφευχθει η διπλή κενή σελίδα μετά τον τίτλο
    {\let\cleardoublepage\relax \frontmatter}
\clearpage
\afterpage{\blankpage}

%Εμπροσθόφυλλο Αγγλικά
\begin{titlepage}


\includegraphics[width=0.6\textwidth]{images/logo}


  \begin{center}
        \vspace*{3cm}
            
        \Huge
        \color{orange}
        \textbf{Thesis Title}\\
         \color{black} 
        \vspace{0.5cm}
        \LARGE
        Subtitle\\
        \vspace{0.5cm}
        
            
        \vspace{1.5cm}
        \LARGE    
        \textbf{Author Name}
            
         \vspace{3.5cm}
        \textbf{Supervisor:} Name
        \vfill    
        \vspace{0.8cm}
            
       
            
        \Large
       KOZANI/MONTH/YEAR
       \end{center}
\end{titlepage}
\afterpage{\blankpage}
\clearpage


% Πνευματικά δικαιώματα
	\subfile{abstract_acknowledgement_preface/plagiarism}
	
% Περίληψη
	\subfile{abstract_acknowledgement_preface/abstract}
	
% Περίληψη αγγλικά
	\subfile{abstract_acknowledgement_preface/abstract_e}	
\clearpage

% Ευχαριστίες
	\subfile{abstract_acknowledgement_preface/acknowledgements}
	
% Πίνακας Περιεχομένων
\renewcommand\contentsname{\LARGE Περιεχόμενα}
\tableofcontents
\addtocontents{toc}{\protect\thispagestyle{fancy}}
\noindent
\clearpage

% Κατάλογος Σχημάτων
\renewcommand{\listfigurename}{\LARGE Κατάλογος Σχημάτων}
	\listoffigures{\protect\thispagestyle{fancy}}
	\noindent \hrulefill
	\clearpage
	
% Κατάλογος Εικόνων
	\listofillustrations{\protect\thispagestyle{fancy}}
	\noindent \hrulefill
	\clearpage

% Κατάλογος Πινάκων
\renewcommand{\listtablename}{ \LARGE Κατάλογος Πινάκων}
	\listoftables{\protect\thispagestyle{fancy}}
	 \noindent \hrulefill
\clearpage	

% Πρόλογος
	\subfile{abstract_acknowledgement_preface/preface}
	
% Εισαγωγή
	\subfile{maintext/chap1}

% Μέρη/Κεφάλαια
	
	

\chapter{Θεωρητικό υπόβαθρο}

\subsection{Υπάρχουσα Βιβλιογραφία}
\subsubsection{Υποενότητα}
Για να τοποθετήσουμε τις αναφορές χρησιμοποιούμε το cite\cite{greekbook},\cite{dcis2011}. Μη ξεχάσετε να εκτελέσετε το bibtex για τη δημιουργία της βιβλιογραφίας.
\clearpage
	\documentclass[../main.tex]{subfiles}
\titleformat{\section}
  {\normalfont\LARGE\bfseries}{\thesection}{1em}{}[{\titlerule[0.8pt]}]
  
 \titleformat{\subsection}
  {\normalfont\Large\bfseries}{\thesubsection}{1em}{}
  
\titleformat{\subsubsection}
  {\normalfont\large\bfseries}{\thesubsubsection}{1em}{}  
  
\begin{document}

    

\section{Ανάλυση και Σχεδίαση Θέματος}

\subsection{ Στόχος και Ερευνητικά Ερωτήματα}
\subsection{ Ανάλυση Μελέτης Περίπτωσης}
\clearpage
\end{document}
    
	
    
\chapter{Υλοποίηση}
\section{Εισαγωγή}
\subsection{Λεπτομέρειες υλοποίησης}


	\include{maintext/chap5}
	

    

\chapter{Συμπεράσματα και Μελλοντική Επέκταση Έργου}

\subsection{Συμπεράσματα}
\subsection{Μελλοντικές Επεκτάσεις}
\clearpage

	

\chapter{Επίλογος}

\section{Απειλές}
\subsection{Σημασία Έργου}
\clearpage


% Παραρτήματα
	%\appendix
    \renewcommand\appendixpagename{Παράρτημα}
    \renewcommand\appendixname{Παράρτημα}
    \begin{appendices}
    
    
	
\chapter{αααα}

Στα παραρτήματα μπορεί να συμπεριληφθούν εργαλεία που χρησιμοποιήθηκαν στην έρευνα ή/και οτιδήποτε άλλο πληροφοριακό υλικό το οποίο δεν είναι δυνατό να ενταχθεί οργανικά στο κυρίως μέρος της μελέτης π.χ. ερωτηματολόγια, κώδικας προγράμματος, φυλλάδια κατασκευαστών, σχήματα κωδικοποίησης  κτλ.

	
% Βιβλιογραφία - Αναφορές
    \renewcommand\bibname{Βιβλιογραφία}
    \bibliographystyle{ieeetr}
	\bibliography{references}
	

	
	
% Συντομογραφίες - Αρκτικόλεξα - Ακρωνύμια &Απόδοση Ξενόγλωσσων όρων

\subfile{appendix/abbreviations}	
	\addcontentsline{toc}{section}{Συντομογραφίες - Αρκτικόλεξα  - Ακρωνύμια}
	\addcontentsline{toc}{section}{Απόδοση Ξενόγλωσσων Όρων}
	%%%%%%%%%%%%%%%%%%%%%%%%%%%%%%%%%%%%%%%%%%%%%
\end{appendices}

\end{document}
